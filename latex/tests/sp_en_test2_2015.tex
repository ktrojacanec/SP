\documentclass[11pt]{examdesign}
%\usepackage{ucs}
\usepackage[T2A]{fontenc}
\usepackage[utf8]{inputenc}
\usepackage{amsmath}
\usepackage{pifont}
\usepackage{verbatim}
\usepackage[ddmmyyyy]{datetime}
\renewcommand{\dateseparator}{.}
\SectionFont{\large\sffamily}
\usepackage[margin=1.5cm]{geometry}
%\Fullpages
\ContinuousNumbering
%\ShortKey
\DefineAnswerWrapper{}{}
\NumberOfVersions{1}
\IncludeFromFile{sp_test2_code.tex}

\def\namedata{Name: \hrulefill \\[5pt]
ID: \hrulefill}

\begin{examtop}
{\parbox{.5\textwidth}{\textbf{\classdata} \\
\examtype, 14.12.2015, Group: \fbox{\textsf{A}}\\ \emph{Each question
has exactly \textbf{one correct} answer.}}
\hfill
\parbox{.45\textwidth}{\normalsize \namedata}}
\end{examtop}

\def\aftersectsep{0pt}
\def\beforesectsep{0pt}
\def\beforeinstsep{0pt}
\def\afterinstsep{0pt}


\class{\Large{Structured Programming}}
\examname{Test 2}
\begin{document}
%\SectionPrefix{Дел \arabic{sectionindex}. \space}


\begin{multiplechoice}[title={},suppressprefix=yes,rearrange=no]

\begin{question}
Which of the following is the correct declaration of twodimensional vector (matrix) of max 30 integers?
    \choice{\texttt{int pole[3*10];}}
    \choice{\texttt{int pole[30];}}
    \choice{\texttt{int pole[6,5];}}
    \choice[!]{\texttt{int pole[5][6];}}
\end{question}

\begin{question}
Given the function prototype: \texttt{float ab(float, float, int);} for some recursive function named `ab', which
of the following answers is a VALID call of the function?
    \choice{\texttt{float x = ab(3.0f, 5.2f); }}
    \choice{\texttt{float x = ab(float 5, float 3, int 8); }}
    \choice[!]{\texttt{float x = ab(3, 5, 7); }}
    \choice{\texttt{float x = float ab(5, 6, 7); }}
\end{question}

\begin{question}
Which of the following declaration is \textbf{not} a valid function prototype:
    \choice{\texttt{int func(char,char);}}
    \choice[!]{\texttt{double funcion;}}
    \choice{\texttt{void func();}}
    \choice{\texttt{float funcion();}}
\end{question}

\begin{question}
What will be the output after execution of the following code segment?
\InsertChunk{c1A}
    \choice{nothing, it's compilation error}
    \choice[!]{\texttt{ime}}
    \choice{\texttt{prezime}}
    \choice{\texttt{trezime}}
\end{question}

\begin{question}
What will be the output after execution of the following code segment?
  \InsertChunk{c2A}
  \choice{\texttt{1}}
  \choice{\texttt{3}}
  \choice[!]{\texttt{5}}
  \choice{can not be determined}
\end{question}

\begin{question}
Which is the correct expression for accessing the last element of the array
defined as: \texttt{int elements[] = \{ 5, 4, 3, 2, 1 \};}
    \choice{\texttt{*(elements + 4 * sizeof(int))}}               
    \choice{\texttt{elements + 5}}
    \choice[!]{\texttt{*(elements + 4)}}
    \choice{\texttt{elements[5]}}  
\end{question}

\begin{question}
What will be the value of \texttt{a} after the execution of the following code segment?
    \InsertChunk{c3A}
    \choice[!]{\texttt{5}}
    \choice{\texttt{4}}
    \choice{\texttt{3}}
    \choice{undefined}
\end{question}
  
\begin{question}
What will be the value of \texttt{len} after the execution of the following code
segment?
    \InsertChunk{c4A}
    \choice[!]{\texttt{5}}
    \choice{\texttt{3}}
    \choice{\texttt{4}}
    \choice{undetermined}
\end{question}

\begin{question}
What will be the output after execution of the following code segment?
\InsertChunk{c6A}
    \choice[!]{\texttt{49 25 9 1 1 9 25 49}}
    \choice{\texttt{49 25 9 1}}
    \choice{\texttt{1 9 25 49}}
    \choice{could not be determined}
\end{question}

\begin{question}
What will be the output after execution of the following code segment?
\InsertChunk{c10}
    \choice{\texttt{1}}
    \choice[!]{\texttt{7}}
    \choice{\texttt{8}}
    \choice{could not be determined}
\end{question}

\end{multiplechoice}

\end{document}
